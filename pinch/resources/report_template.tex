\documentclass[12pt, letterpaper]{article}

% Set document data
\author{pinch-analyis package}
\title{Pinch Point Analysis}

% Math environments
\usepackage{amsmath}
\usepackage{amsfonts}
\usepackage{amssymb}
\usepackage{amsthm}

% Chemical formulas
\usepackage[version=4]{mhchem}
\usepackage{chemfig}

% SI units and friends
\usepackage{siunitx}
\DeclareSIUnit\Fahrenheit{\degree~F}
\DeclareSIUnit\degF{\degree F}
\DeclareSIUnit\degC{\degree C}
\DeclareSIUnit\pound{lb}
\DeclareSIUnit\poundmass{lb\textsubscript{m}}
\DeclareSIUnit\foot{ft}
\DeclareSIUnit\atm{atm}
\DeclareSIUnit\poise{P}
\DeclareSIUnit\year{year}
\DeclareSIUnit\btu{btu}

% Page layout
\usepackage[margin=1in,paper=letterpaper]{geometry}

% Fonts
\usepackage{stix}

% Fancy linked references
\usepackage{varioref}

% Hyperlinks
% These will turn internal cross-references, in-text citations,
% and headings into clickable links in the PDF. The link text
% will still be black.
\usepackage{color}
\usepackage[%
	pdfborder=0,
	colorlinks=true,
	urlcolor=black,
	linkcolor=black,
	citecolor=black]{hyperref}

% Figures and captions
\usepackage{graphicx}
\usepackage{booktabs}
\usepackage[margin=1cm]{caption}

\begin{document}
\maketitle
\tableofcontents

\section{Tables}

\subsection{Problem Design Table}

\begin{table}[h!btp]
	\centering
	\caption{Problem Design Table}
	\input{\VAR{contents.get("tables").get("problem design table")}}
\end{table}

\section{plots}

\subsection{Composite Plots}

\begin{figure}[h!btp]
	\centering
	\caption{Cold Composite Plot}
	\input{\VAR{contents.get("plots").get("cold composite")}}
\end{figure}

\begin{figure}[h!btp]
	\centering
	\caption{Combined Composite Plot}
	\input{\VAR{contents.get("plots").get("combined composite")}}
\end{figure}

\begin{figure}[h!btp]
	\centering
	\caption{Grand Composite Plot}
	\input{\VAR{contents.get("plots").get("grand composite")}}
\end{figure}

\begin{figure}[h!btp]
	\centering
	\caption{Hot Composite Plot}
	\input{\VAR{contents.get("plots").get("hot composite")}}
\end{figure}


\end{document}
